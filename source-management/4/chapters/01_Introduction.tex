%\textcolor{blue}{Sabine}}
Today, the world is facing multiple crises with different social, economic, and ecological consequences. Recent events like the Covid-19 pandemic and the Russia-Ukraine War have highlighted the interdependencies of global supply chains and economic value networks. 

Challenges such as climate change, supply chain disruptions, and healthcare availability, define a new era where "managing disruptions defines sustainable growth more than managing continuity" \cite{McKinsey2022}. Economic adversities can occur at any time and in various granularities, such as company crises, market crises, or global economic crises. To effectively address these challenges, businesses must continually adapt their operating models, value chains, and global networks to improve their flexibility and ability to respond quickly and agilely to changing environmental factors. This is encapsulated by the concept of resilience.

Resilience research is a challenging but urgently needed scientific field which will contribute to solving urgent societal issues. In response to this urgent need, researchers in various fields, including information and communications technology, data science, and artificial intelligence~(compare~e.g.,~\cite{Palen2007},~\cite{Alshaikh2012},~\cite{Soden2014},~\cite{Soden2018}), have made significant contributions to resilience and crisis research. 

The CoyPu project \cite{CoyPu} aims to increase the transparency of value chains and the understanding of complex mechanisms of crisis factors at a global scale by using semantically represented data and AI analytics. Through a large consortium of partners, the project integrates, models, and analyzes huge amounts of data to build a new basis for situational awareness and decision making, as well as for the elaboration of advanced resilience strategies. In the context of a future CoyPu platform, semantic technologies such as RDF, OWL, and SPARQL combine data interoperability and "cross-silo" queries with decentralized storage. The CoyPu Knowledge Graph provides macro-economically relevant and market-specific data, as well as information on current global crisis and conflict events, which can be integrated with external data on an ad-hoc basis. This paper focuses on the subset of trade-related policy measures, sanctions, and political violence and conflicts within the CoyPu Knowledge Graph.

Temporal Knowledge Graphs (tKG) are Knowledge Graphs (KG) where facts occur, recur, or evolve over time \cite{Trivedi2017}. Triples are extended with timestamps to indicate that they are valid at a given time, allowing to hold time-evolving multi-relational data \cite{Han2021xerte}. Because they are not only able to represent the interconnectivity of systems, but also their dynamic evolvement, tKG are highly suitable for application in crisis and resilience research. They can be used to understand the evolution of complex economic supply chains over time, with a particular focus on the impact of interlinked crises. The research field of tKG forecasting predicts facts at future timesteps based on a history of a KG [3]. In crisis and resilience research this capability can be applied not only to analyse the interconnectivity of systems, but also to predict the future evolvement and links in these systems, allowing for timely interventions.

This paper introduces a novel temporal Knowledge Graph that covers interlinked worldwide crisis and trade sanction events for the year 2021, providing a comprehensive view of the dynamic relationships of these events.

\subsection{Use Case}\label{sec:UseCase}
By using the presented tKG for downstream analysis and learning tasks, we can identify patterns and predict future developments in the global landscape of crisis and trade sanctions. Specifically, we propose the use of tKG forecasting (see e.g.,
\cite{Jin2019oldrenet}, \cite{Li2021regcn}, \cite{Han2021xerte}) to predict upcoming global trade alert events and their links based on previous crisis events. This research opens up new possibilities for understanding the complex interactions between global crises and trade sanctions and lays the groundwork for future studies in this field.

\subsection{Structure and Contribution}
This paper first provides an overview of existing work on KG and vocabularies for resilience research and prevalent tKG datasets (Section~\ref{sec:rel_work}), along with an overview of the utilized dataset resources (Section~\ref{sec:resources}). Further, we describe our approach for creating the tKG (Section~\ref{sec:method}). To understand the properties of the created tKG, we perform a technical analysis and visualize a selected graph snapshot, providing insights for tKG Forecasting (Section~\ref{sec:experiments}). We conclude with an outlook describing the potential usage of this tKG in further research, as well as specifically in the CoyPu project (Section~\ref{sec:conclusion}). We publish the tKG dataset and associated code for creating and analysing the tKG\footnote{\url{https://github.com/GastJulia/TKG-ACLED-GTA-Dataset}}. 

% In this paper we make the following contributions:
% \begin{enumerate}
%     \item A comprehensive description of the process to retrieve Knowledge Graphs for resilience research from public data.
%     \item The curation of a temporal Knowledge Graph for time-aware resilience research and prediction from two data sources.
%     \item An extensive analysis of the created tKG.
%     \item Publication of the tKG dataset and with associated code for creating and and analysing the tKG\footnote{\url{https://github.com/GastJulia/TKG-ACLED-GTA-Dataset} \textcolor{red}{TODO: upload code and dataset}}.
% \end{enumerate}
