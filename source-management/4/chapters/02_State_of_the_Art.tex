\subsection{Knowledge Graphs and Vocabularies for Resilience Research}
The use of Knowledge Graphs in resilience and crisis research has gained increasing attention in recent years \cite{Kim2022}. KG offer a flexible and comprehensive approach to modeling and analysing complex systems \cite{TILLY2021115765}, making them suitable for a wide range of domains, including macro-economical analysis \cite{Yang2020}, which is a main focus of the CoyPu research project.

Creating a KG requires a structured and standardized way to represent data in a machine-readable format. Ontologies offer a means to provide a shared vocabulary of terms and concepts that enable data to be integrated and analysed in a consistent and interoperable way. Although there exist established vocabularies \cite{Crofts2011} to model events, including their relevant actors, occurrence, locality, and other significant properties, the reuse of such vocabularies presents several challenges. These challenges arise from the complex, highly domain-specific nature of these vocabularies, divergent levels of granularity, lack of easy extensibility, and the difficulty of creating interoperable mappings between different ontologies. As a result, in the CoyPu project a custom ontology - the CoyPu COY ontology \cite{coy_2023} - was developed to model the KG.

\subsection{Temporal Knowledge Graph Datasets}  
\citet{Zhang2021} provide a comprehensive overview of existing temporal RDF models. We follow the work of \citet{Trivedi2017}, \citet{Li2021regcn}, \citet{Han2021xerte}, and others, who represent tKG as sequences of timestamped KG. A timestamped KG, or KG snapshot, denoted as $G_t = \{V,R, \mathcal{E}_t\}$, captures the state of the tKG at a specific timestep $t$, where $V$ is the set of entities, $R$ is the set of relations, and $\mathcal{E}_t$ is the set of quadruples \cite{Li2021regcn}. A quadruple consists of four elements, such as \textit{(Event A, hasActor, French Police Forces, 2021-07-01)}. 

In the domain of tKG analysis, six datasets have been published and utilized, including different versions of the Integrated Crisis and Early Warning System (ICEWS) \cite{Boschee2015}: ICEWS05-15 \cite{GarciaDuran2018ICEWS14}, ICEWS14 \cite{GarciaDuran2018ICEWS14}, and ICEWS18 \cite{Jin2019oldrenet} (the numbers describe the respective years); GDELT \cite{Leetaru2013GDELT}; YAGO \cite{Mahdisoltani2015YAGO}; and WIKI \cite{Leblay2018WIKI} (preprocessed by \citet{Jin2019oldrenet}). Notably, the three versions of ICEWS cover the crisis topic, demonstrating the applicability of tKG to crisis research. However, to the best of our knowledge, no tKG currently exists that describes trade relations and sanctions over time. Additionally, none of the existing tKG merge data from multiple sources to provide a comprehensive view or analyse the interconnection of different event types. Finally, to our knowledge, no other study has analysed the evolution of graph properties over time for tKG.
