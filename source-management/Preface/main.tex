\documentclass{ceurart}

%%
%% One can fix some overfulls
\sloppy

\begin{document}

%%
%% Rights management information.
%% CC-BY is default license.
\copyrightyear{2023}
\copyrightclause{Copyright for this paper by its authors.
  Use permitted under Creative Commons License Attribution 4.0
  International (CC BY 4.0).}

%%
%% This command is for the conference information
\conference{Second International Workshop on Linked Data-driven Resilience Research (D2R2'23) co-located with ESWC 2023, May 28th, 2023, Hersonissos, Greece}

\title{Second International Workshop on\\ Linked Data-driven Resilience Research 2023}

%%
%% The "author" command and its associated commands are used to define
%% the authors and their affiliations.

\author[1]{Sebastian Tramp}[%
orcid=0000-0003-4707-2864,
email=sebastian.tramp@eccenca.com,
url=http://aksw.org/SebastianTramp,
]

\author[2]{Ricardo Usbeck}[%
orcid=0000-0002-0191-7211,
email=ricardo.usbeck@uni-hamburg.de,
url=http://aksw.org/RicardoUsbeck,
]
\author[1]{Natanael Arndt}[%
orcid=0000-0002-8130-8677,
email=natanael.arndt@eccenca.com,
url=http://aksw.org/NatanaelArndt,
]

\author[3]{Julia Holze}[%
email=holze@infai.org,
url=http://aksw.org/JuliaHolze,
]

\author[4]{Sören Auer}[%
orcid=0000-0002-0698-2864,
email=auer@tib.eu,
url=http://aksw.org/SoerenAuer,
]

\address[1]{eccenca GmbH, Hainstraße 8, 04109 Leipzig, Germany}

\address[2]{University of Hamburg}

\address[3]{Institut für Angewandte Informatik e.V., Goerdelerring 9, 04109 Leipzig, Germany}

\address[4]{Gottfried Wilhelm Leibniz University Hannover and Technische Informationsbibliothek (TIB)}


%%
%% The abstract is a short summary of the work to be presented in the
%% article.
\begin{abstract}
In the face of continuously changing contextual conditions and ubiquitous disruptive crisis events, the concept of resilience refers to some of the most urgent, challenging, and interesting issues of nowadays society. Economic value networks, technical infrastructures, health systems, and social textures alike need to unfold capacities to withstand, adapt, recover, or even refine and transform themselves to stay ahead of changes.
\end{abstract}

%%
%% Keywords. The author(s) should pick words that accurately describe
%% the work being presented. Separate the keywords with commas.
\begin{keywords}
  Crisis Information \sep
  Resilience \sep
  Knowledge Graphs \sep
  Linked Data \sep
  Semantic Web \sep
  CoyPu
\end{keywords}

%%
%% This command processes the author and affiliation and title
%% information and builds the first part of the formatted document.
\maketitle

%\section{Introduction}

The D2R2’23 workshop (\url{https://d2r2.aksw.org/}), which is organized by the CoyPu project (\url{https://coypu.org/}), provides an open forum to exchange current issues, ideas, and trends in the area of Data-driven Resilience Research among scientists, software engineers, resilience practitioners, and domain experts. Ongoing technological developments, current research approaches as well as use case scenarios, and field reports are presented and discussed with a broad and multi-disciplinary specialist audience. We have received contributions of novel results, ongoing work, and position papers focusing on various aspects of Data-driven Resilience Research from a scientific or practical perspective.

The workshop is held during the Extended Semantic Web Conference (ESWC) in Crete, Greece.
The co-location of the workshop with ESWC allows a wide exchange among the community of Semantic Web experts, data scientists, knowledge engineers, and further interested parties.

For the workshop, we received a total of 12 contributions, out of which 7 were selected to be included in the proceedings. These are 6 full papers and 1 short paper.

The D2R2 workshop's first issue~\cite{D2R2-22} was held at the Data Week Leipzig 2022.

\section*{Program Committee}

% updated 2023-05-04

\begin{itemize}
    \item Allard Oelen, Technische Informationsbibliothek (TIB)
    \item Angelie Kraft, University of Hamburg
    \item Antonin Delpeuch, Institute for Applied Informatics
    \item Cedric Möller, University of Hamburg
    \item Edgard Marx, Senior Linked Data Expert, eccenca GmbH and HTWK Leipzig
    \item Eva Hoerster, Senior Data Scientist, DATEV eG
    \item Felix Engel, Technische Informationsbibliothek (TIB)
    \item Julia Gastinger, NEC Laboratories Europe and University of Mannheim
    \item Junbo Huang, University of Hamburg
    \item Lars-Peter Meyer,  Institute for Applied Informatics
    \item Magnus Knuth, Senior Linked Data Expert, eccenca GmbH
    \item Maria-Esther Vidal, Technische Informationsbibliothek (TIB)
    \item Marvin Hofer, Leipzig University
    \item Michael Martin, Head of Competence Center, Institute for Applied Informatics
    \item Natanael Arndt, Senior Linked Data Expert, eccenca GmbH
    \item Nenad Krdavac, Technische Informationsbibliothek (TIB)
    \item Norman Radtke, Institute for Applied Informatics
    \item Patrick Westphal, University of Hamburg
    \item Ricardo Usbeck, University of Hamburg
    \item Sabine Gründer-Fahrer, Institute for Applied Informatics
    \item Sebastian Tramp, CTO, eccenca GmbH
    \item Simon Bin, Institute for Applied Informatics
    \item Sören Auer, Gottfried Wilhelm Leibniz University Hannover and Technische Informationsbibliothek (TIB)
\end{itemize}


\begin{acknowledgments}
We want to thank all contributors and the whole program committee for their work.
This work has been supported by the German Federal Ministry for Economic Affairs and Climate Action (BMWK) as part of the project CoyPu under grant number 01MK21007[A-L].
\end{acknowledgments}


%%
%% Define the bibliography file to be used
\bibliography{references}

%%
%% If your work has an appendix, this is the place to put it.

\end{document}

%%
%% End of file
